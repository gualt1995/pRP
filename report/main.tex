%-----------------------------------------------
% DOCUMENT PACKAGES
%-----------------------------------------------
\documentclass[10pt]{article}
\usepackage[utf8]{inputenc}
\usepackage[T1]{fontenc}
\usepackage{graphicx}
\usepackage[margin=1.3in]{geometry}
\usepackage{hyperref}
\usepackage[french]{babel}
\usepackage[small, sc, bf, center]{titlesec}
\usepackage{listings}
\usepackage{amsmath, amssymb, mathtools}
\usepackage{xcolor}
\usepackage{fancyhdr}
\usepackage{tikz}
\usepackage{tkz-graph}
%\usepackage{subcaption}
%\usepackage{multicol}
%-----------------------------------------------
% DOCUMENT CONFIG
%-----------------------------------------------

% Add point after title number
\titleformat{\section}[block]{\sc\bfseries\center\large}{\thesection.}{0.5em}{}
\titleformat{\subsection}[block]{\sc\bfseries\center}{\thesubsection.}{0.5em}{}
\titleformat{\subsubsection}[block]{\sc\bfseries\center}{\thesubsubsection.}{0.5em}{}
% Page number reformat
\pagestyle{fancy}
\fancyfoot[C]{--~\thepage~--}
% Deactivate fancyhdr header
\renewcommand{\headrulewidth}{0pt}
\fancyhead{}
% tikz
\tikzstyle{vertex}=[circle, draw, inner sep=0pt, minimum size=6pt]
\newcommand{\vertex}{\node[vertex]}
\usetikzlibrary{arrows,petri,topaths,calc}

%-----------------------------------------------
% DOCUMENT BODY
%-----------------------------------------------
\begin{document}
	
\begin{center}
	\textbf{Projet de résolution de problèmes\\[.5cm]Metaheuristiques pour la résolution du\\problème de l'arbre de
	Steiner de poids minimum}\\[.5cm]
	\textit{Alexandre Bontems, Gualtiero Mottola}\\
	
\end{center}

\tableofcontents

\section{Introduction}
	Ce projet aborde la résolution du problème de l'arbre de Steiner de poids minimum avec deux méthodes: un algorithme génétique et un algorithme de recherche locale. Ce problème d'optimisation combinatoire 
	
	\begin{figure}[h!]
		\centering
		\begin{tikzpicture}[scale=0.75, transform shape]
			\tikzstyle{edgesol}=[draw=blue!50]
			\tikzstyle{termvertex}=[fill=black]
			% Noeuds terminaux
			\vertex[termvertex] (v1) at (-1,3) [label=above:$v_1$] {};
			\vertex[termvertex] (v2) at (4,3) [label=above:$v_2$] {};									\vertex[termvertex] (v3) at (-0.5,0) [label=below:$v_3$] {};
			\vertex[termvertex] (v4) at (4,-1) [label=below:$v_4$] {};
			\vertex[termvertex] (v5) at (6,-.5) [label=right:$v_5$] {};
			% Noeuds de steiner
			\vertex (u1) at (1.5,2) [label=left:$u_1$] {};
			\vertex (u2) at (3,1) [label=right:$u_2$] {};
			\vertex (u3) at (-2.5,1.5) [label=left:$u_3$] {};
			\vertex (u4) at (6.5,1.5) [label=right:$u_4$] {};
			
			\path
			% Solution
			(v1) edge[edgesol] node[fill=white]{$2$} (u1)
			(u1) edge[edgesol] node[fill=white]{$1$} (u2)
			(v3) edge[edgesol] node[fill=white]{$4$} (u2)
			(u1) edge[edgesol] node[fill=white]{$2$} (v2)
			(v2) edge[edgesol] node[fill=white]{$5$} (v5)
			% Reste des arcs
			(v1) edge node[fill=white]{$2$} (u3)
			(u3) edge node[fill=white]{$8$} (v3)
			(v1) edge node[fill=white]{$9$} (v3)
			(v3) edge node[fill=white]{$8$} (v4)
			(u2) edge node[fill=white]{$3$} (v4)
			(v1) edge node[fill=white]{$8$} (v2)
			(v2) edge node[fill=white]{$8$} (u4)
			(u4) edge node[fill=white]{$8$} (v5)
			(u2) edge node[fill=white]{$5$} (v5)
			;
		\end{tikzpicture}
		\caption{Exemple de Steiner Tree Problem}
	\end{figure}
	
	L'algorithme génétique a été conçu de façon modulaire pour pouvoir accueillir différentes fonctions d'initialisation, différents opérateurs de croisement et de mutation, etc. Il a ainsi été possible d'étudier les performances de chaque composante pour les différentes types d'instances considérés. L'algorithme de recherche locale [...]
	
\section{Algorithme génétique}
	Un algorithme génétique consiste généralement en l'enchaînement des actions suivantes sur plusieurs générations: \textit{Initialisation, Sélection, Crossover, Mutation}. La phase d'initialisation permet de générer une population d'individus qui seront ensuite sélectionnés, croisés, mutés afin de construire la génération suivante.

	\subsection{Population initiale}
		\subsubsection{Initialisation aléatoire}
			
		\subsubsection{Heuristique du plus court chemin}
		\subsubsection{Heuristique de l'arbre couvrant minimum}

	\subsection{Opérateurs de sélection et croisement}
			
		\subsubsection{Fitness proportionate selection}
		\subsubsection{Stochastic Universal Sampling}
		\subsubsection{Tournament Selection}
	
	
\section{Recherche locale}
\section{Conclusion}
	
\end{document}